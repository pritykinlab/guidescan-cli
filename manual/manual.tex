\documentclass[11pt]{article}

\usepackage[utf8]{inputenc}
\usepackage{minted}
\usepackage{geometry}
\usepackage{xcolor}
\usepackage{color,soul}
\usepackage{parskip}
\usepackage{amsmath}
\usepackage{amsfonts}
\usepackage{graphicx}
\usepackage{enumitem}
\usepackage{placeins}
\usepackage{varwidth}
\usepackage[labelfont=bf]{caption}
\usepackage{booktabs} % To thicken table lines

\usepackage[citestyle=alphabetic,bibstyle=numeric]{biblatex}
\addbibresource{manual.bib}

\usepackage{fancyhdr}
\usepackage{hyperref}

\fancyhead{} % Clear the headers
\renewcommand{\headrulewidth}{0pt} % Width of line at top of page
\fancyhead[L]{\slshape\leftmark} % Mark right [R] of page with Chapter name [\leftmark]

\pagestyle{fancy}

\linespread{1.2}

\geometry{margin=1in}

\newcommand{\CC}{C\nolinebreak\hspace{-.05em}\raisebox{.4ex}{\tiny\bf
    +}\nolinebreak\hspace{-.10em}\raisebox{.4ex}{\tiny\bf +}}
\def\CC{{C\nolinebreak[4]\hspace{-.05em}\raisebox{.4ex}{\tiny\bf ++}}}

\author{Henri Schmidt$^1$}
\title{Guidescan2 Software Manual}
\date{
    $^1$Department of Computer Science, Princeton University\\[2ex]%
    \today
}

\begin{document}

\maketitle
\tableofcontents
\newpage

\section{Installation}
\subsection{Software Dependencies}
The off-target enumeration tool, Guidescan2, relies on the
installation of a modern \CC\ compiler and associated
buildtools. Explicitly, the tool has the following dependencies.
\begin{itemize}
\item CMake version 3.1.0 or higher
\item \CC\ compiler that supports C++11 features such as:
  \begin{itemize}
  \item regex
  \item constexpr
  \item default constructors
  \end{itemize}
\item \CC\ support for POSIX threads (pthreads) 
\end{itemize}
It is known that GCC version 4.9 or greater and clang 3.1 or greater
with libc\texttt{++} are both supported. But to error on the side of caution,
the most up to date and stable version of GCC is recommended. As often
high-performance computing clusters do not possess recent versions of
\CC\ compilers, it is therefore recommended that the software is built
and executed in some sort of virtual environment. We have had success
with Singularity and Docker for these special cases, though if a
modern compiler is available, that can be used instead.

The aforementioned dependencies are the bare minimum to obtain
anything useful out of the tool. However, it is strongly recommended,
and necessary in some our pipelines, that biological computing tools
are also available. In particular, the Samtools suite is useful since
the output database will be in a SAM file format
\parencite{danecek2021twelve}. There are also bindings for the tool in
several programming languages that may help users to interface with
the Guidescan2 databases programatically.

Helper scripts for various pipelines require a Python3 and Conda
installation. We include a Conda environment file that contains all
the necessary dependencies, though the scripts should run fine if
standard data science packages are installed. 

\subsection{Build Procedure}
Building the off-target enumeration tool, Guidescan2, is
straightforward usign CMake once the dependencies have been
installed. First set the working directory to the guidescan-cli
project root. Then execute the following commands.

\begin{minted}[frame=lines]{shell-session}
  $ mkdir build; cd build
  $ cmake -DCMAKE_BUILD_TYPE=RELEASE ..
  $ make
\end{minted}

The executable will be output in the \texttt{build/bin/} directory
under the name \texttt{guidescan}. To make things convenient, we
recommend that you add the program to your path. One way to do this is
to place the executable under \texttt{\$HOME/bin/} and then append
this directory to your
\texttt{\$PATH}. The following code snippet does this and then
permanently updates your \texttt{\$PATH} in your bashrc.

\begin{minted}[frame=lines]{shell-session}
  $ mkdir -p $HOME/bin
  $ cp bin/guidescan $HOME/bin
  $ echo export PATH=$PATH:$HOME/bin >> ~/.bashrc
  $ source ~/.bashrc
\end{minted}

\section{Command Line Interface}
For ease of exposition, we will assume Guidescan2 is installed under
your path and can be executed by simply running the command
$\texttt{guidescan}$. However, everything will apply even if you
execute Guidescan2 using its absolute path.

To run the Guidescan2 off-target enumeration and view its
sub-commands, simply execute the program with the \texttt{--help}
flag.
\begin{minted}[frame=lines]{shell-session}
  $ guidescan --help
\end{minted}

There are three sub-commands for the Guidescan2 tool. 
\begin{itemize}
\item \texttt{index}: Constructs a genomic index from a FASTA file.
\item \texttt{enumerate}: Enumerates off-targets against a reference
  genomic index for a particular set of kmers. 
\end{itemize}
The sub-commands each have their own interfaces, accessed as such:
\begin{minted}[frame=lines]{shell-session}
  $ guidescan [SUBCOMMAND] --help
\end{minted}

\subsection{Genome Indexing}

\begin{minted}[frame=lines]{shell-session}
  $ guidescan index --help
\end{minted}

The \texttt{index} sub-command takes a FASTA file as input and
constructs a compressed genomic index that is used for off-target
search. By default, the index is stored as three files in the same
directory as the FASTA file with the name \texttt{*.index.*}. The
indices consist of a forward and reverse strand index as well as a
small metadata file containing chromosome information. Auxiliary files
with extensions \texttt{.forward.fna} and \texttt{.reverse.fna} are
written to disk during the indexing process but these can be deleted
once the genome is constructed.

The \texttt{index} step is the only step that requires a decent amount
of memory to execute. For the human reference genome, 32GB of memory
will suffice. However once the indices are constructed they can be
transferred across devices as they are relatively small. Indices for a
wide variety of genomes are available at
\texttt{https://guidescan.com/downloads}. And though this steps
requires a moderate amount of memory, it is quick to execute, taking
under thirty minutes on a Lenovo Thinkpad t490 with 32GB of RAM and an
8-core processor.

\subsection{Off-Target Enumeration}
\begin{minted}[frame=lines]{shell-session}
  $ guidescan enumerate --help
\end{minted}

\begin{figure}[!htb]
  \begin{tabular}{|l|l|}
    \hline
    Option            & Description                                                                                                                                                                                                                                                                                         \\ \hline
    \texttt{--start}           & \begin{tabular}[c]{@{}l@{}}Specify whether the PAM should be matched at the start or end of the\\ protospacer sequence. The default is FALSE, in which the PAM is at\\ the 3' end of the protospacer.\end{tabular}                                                                                  \\ \hline
    \texttt{--max-off-targets} & \begin{tabular}[c]{@{}l@{}}Maximum number of off-targets to store at each mismatch distance. If \\ this takes a negative value, all off-targets are enumerated at that distance.\\ The default value is -1, in which all off-targets are enumerated.\end{tabular}                                   \\ \hline
    \texttt{-n, --threads}     & \begin{tabular}[c]{@{}l@{}}The number of threads to use in finding off-targets. The kmers file will\\ be split into pieces and each piece will be enumerated across different \\ threads. The default value is the number of threads the machine \\ possesses.\end{tabular}                         \\ \hline
    \texttt{-a, --alt-pam}     & \begin{tabular}[c]{@{}l@{}}The set of alternative PAMs used to find off-targets. That is, if there exists\\ an off-target with one of the specified alternative PAMs, it will be output. \\ Input is given as a comma  separated list of PAMs. The default value is the \\ empty list.\end{tabular} \\ \hline
    \texttt{-m, --mismatches}  & \begin{tabular}[c]{@{}l@{}}The maximum distance from the input sequence an off-target can have. \\ The default value is 3.\end{tabular}                                                                                                                                                             \\ \hline
    \texttt{--rna-bulges}      & \begin{tabular}[c]{@{}l@{}}The number of RNA bulges to allow in the protospacer when finding\\ off-targets. The default value is 0 as this can greatly increase the \\ running time.\end{tabular}                                                                                                   \\ \hline
    \texttt{--dna-bulges}      & \begin{tabular}[c]{@{}l@{}}The number of DNA bulges to allow in the off-target when finding\\ off-targets. The default value is 0 as this can greatly increase the \\ running time.\end{tabular}                                                                                                    \\ \hline
    \texttt{-t, --threshold}   & \begin{tabular}[c]{@{}l@{}}Removes sequences from output with off-targets at a distance at\\ or below this threshold. If this takes a negative value, no thresholding\\ of off-targets is performed and all sequences are included in output.\\ The default value is 1.\end{tabular}                \\ \hline
    \texttt{--format}          & The format of the output file. The default value is csv.                                                                                                                                                                                                                                            \\ \hline
    \texttt{-f, --kmers-file}  & \begin{tabular}[c]{@{}l@{}}The file containing the set of kmers to enumerate off-targets against. This\\ is a CSV file format that is described in the manual. This option is required.\end{tabular}                                                                                                \\ \hline
    \texttt{-o, --output}      & The name of the output file. This option is required.                                                                                                                                                                                                                                               \\ \hline
  \end{tabular}
  \caption{\label{fig:enumerate_options} Options for the \texttt{enumerate}
    sub-command.}
\end{figure}


Once an index is constructed, off-target enumeration proceeds in an
online fashion with the \texttt{enumerate} sub-command. Input to this
command is an index and a set of kmers to evaluate off-targets
against. PAMs are included in this kmer set and are matched at the
specified end of the kmer. The output is a database in SAM file format
containing kmers that have passed the Guidescan2 filtering
step. Complete off-target information is hex-encoded in the
\texttt{of} field and can be decoded using an included Python3 script
described later.

The command understands several options that are described fully in
(Figure \ref{fig:enumerate_options}). Most importantly is the flag
\texttt{-m/--mismatches} that describes the mismatch radius to search
for kmer matches within. Typically, this parameter is set to a small
value of, typically at most 4 or 5, as the search complexity grows
exponentially with this parameter. Second, it is often useful to
specify an alternative set of PAMs to match off-targets against since
different CRISPR systems have varying PAM specificity. Of course this
step can be emulated using repeated calls to \texttt{enumerate} with
different PAMs, but we include this as a feature for
convenience. Multi-threading across cores is available with the
\texttt{-n/--threads} option. This may result in out of order entries
compared to the input kmer set since threads may complete off-target
enumeration at different speeds.

The enumerate command {\bf requires} a set of genomic sequences to
evaluate, we refer to these as the {\it kmer set}. The kmer set is
specified in a very flexible CSV format described below. It can be
automatically generated by the tool or manually specified. The kmer
set consists of six columns:

\begin{center}
  \texttt{id,sequence,pam,chromosome,position,sense} 
\end{center}

of which only \texttt{id}, \texttt{sequence} and \texttt{sense} are
required, though all must be specified in the header. An example is
shown in (Figure \ref{fig:kmer_set}). Notice that for the third
example, we don't specify a PAM. In this case, we simply match
off-targets against the kmer with no regards to PAM specificity ---
off-targets don't need a PAM.

\begin{figure}[ht]
  \centering
  \begin{minted}[frame=lines]{text}
id,sequence,pam,chromosome,position,sense
ce11_example_1,GCCGTTCAGGAGCTCGACGA,NGG,chrIV,5499,-
ce11_example_2,CAAAATATGAAATTTTCAAG,NGG,chrIV,23896,+
ce11_example_3,TCTACTGAAAGTTTGCAAAA,,chrIV,5499,-
ce11_example_4,TATAAACTGTCAAAGTTGAG,NGG,chrIV,23703,+
  \end{minted}
  \caption{\label{fig:kmer_set}: An example kmer file for the
    Caenorhabditis elegans genome.}
\end{figure}

\subsection{Kmer Set Generation}
As mentioned earlier, {\it kmer set} generation can be peformed manually,
for example, when you want to evaluate a specific set of gRNAs.
Additionally, Guidescan2 can also generate {\it kmer set} automatically.
To generate {\it kmer sets} we include a Python3 script,
\texttt{generate\_kmers.py}, that generates a set of kmers against an
input FASTA file. To use the script simply write
\begin{minted}[frame=lines]{shell-session}
  $ python generate_kmers.py FASTA_FILE 
\end{minted}
and the output CSV will be sent directly to \texttt{stdout}.

The script understands several options, though it is only mandatory to
specify the FASTA file with which kmers are scanned. Additionally, one
can specify PAM and kmer length, with defaults set to NGG and 20. By
default kmers are only generated from chromosomes that are
sufficiently long, but this can be specified using the
\texttt{--min-chr-length} option. It also uses positional information
as an identifier, but an additional prefix can be specified using the
\texttt{--prefix} option.

The dependencies for the script are Python3 and the Biopython toolkit.

\section{Off-Target Databases}
Guidescan2 outputs databases in the SAM file format, a common file
format used in Bioinformatics. At minimum, the Guidescan2 database
contains the off-target set for each screened kmer as well as
chromosomal structure information. Importantly, \textbf{the header is
  essential} for downstream processing; take care to not delete it
accidently. The additional fields position, chromosome, and strand are
filled in only if they are specified in the {\it kmer set} for the
corresponding kmer. 

Though convenient for storage, compression, and a variety of
bioinformatics tasks, the off-target information is difficult to pull
out. Namely, off-target information is hex-encoded in the attributes
field with label \texttt{of} and is not human-readable. As such, we
include a script to decode the off-target information into
human-readable format.

\subsection{Decoding}

The script \texttt{decode\_kmers.py} is a Python3 script that decodes
Guidescan2 databases sequentially into a human-readable CSV format. It
takes the Guidescan2 SAM/BAM database and the original FASTA file as
input and outputs a CSV to \texttt{stdout}. The script outputs
information in two modes, denoted {\it succinct} and {\it
  complete}, which we will describe below.

For most use cases {\it succinct} mode will be sufficient. In this
mode each kmer's off-targets are decoded and the information is
summarized into multiple columns. Namely, we ouput the counts for
matches at various distances and an overall score denoted specificity
which is defined in the Supplementary Text. We note that the
specificity is only defined for 20-mers with NGG PAMs; this column of
the output will be empty otherwise. We also include additional
metadata with obvious meaning upon examination.

In {\it complete} mode we explicitly write out the sequence and
genomic location of all off-targets for each kmer. In addition, in the
20-mer case we include the CFD score from which our specificity score
is derived. We warn that since some kmers can have up to hundreds of
thousands of off-targets, {\it complete} mode can result in an
extremely large output. If this is of concern we recommend that you
re-run Guidescan2 with a smaller value of mismatch parameter and then
decode the databases again.

To run the script simply write,
\begin{minted}[frame=lines]{shell-session}
  $ python scripts/decode_database.py GUIDESCAN_SAM_DB FASTA_FILE 
\end{minted}
The script \textbf{depends} on the two pickled files in the
subdirectory \texttt{scripts/cfd/}. As such, it is required that the
script and the sub-directory \texttt{scripts/cfd/} are in the location
for the program to properly execute. By default this is the
case, but we caution that when moving the script, move the
\texttt{scripts/cfd/} sub-directory as well.

\section{Guidescan2 Pipelines}
\subsection{Analysis of Individual gRNAs}
Here we introduce a pipeline for the analysis of a small set of gRNAs
that have been gathered from external sources. In our example, we will
analyze a subset of genes targeted in a CRISPRko essentiality screen
\parencite{Wang1096}.  The files for this example can be found in the
aforementioned publication, but are included as CSV files in the
\texttt{examples/} sub-directory. The two files we will use contain
the set of gRNAs used in the knockout screen
(\texttt{sabatini\_grnas.csv}) and gRNA abundance after harvesting at
initial and final time points (\texttt{sabatini\_read\_counts.csv}).

To start we will index the reference genome to analyze off-targets
against. Since the Sabatini2015 screen is performed in human cells,
the human reference genome \textit{hg38} is a natural choice. We will
use the RefSeq version of this reference genome which is provided by
the National Center for Biotechnology Information at the address:

\begin{center}
\texttt{https://www.ncbi.nlm.nih.gov/assembly/GCF\_000001405.26/}
\end{center}

Once downloaded the reference genome must be decompressed into a raw
FASTA file.  To do this one can run the following sequence of
commands, assuming the downloaded tarball has name \texttt{hg38.tar}.

\begin{minted}[frame=lines]{shell-session}
  $ tar -xvf hg38.tar
  $ cd ncbi-genomes-2022-tarball
  $ gunzip GCF_000001405.26_GRCh38_genomic.fna.gz
\end{minted}

It will also be useful to include the file \texttt{chr2acc} which maps
chromosome names to NCBI accessions, which are used internally in
Guidescan2. For example, this file can be found on the NCBI FTP server
at the location:

\begin{center}
\texttt{.../Primary\_Assembly/assembled\_chromosomes/chr2acc}
\end{center}

and it takes the form in (Figure \ref{fig:chr2acc}).

\begin{figure}[t]
  \centering
  \begin{minted}[frame=lines]{text}
#Chromosome	Accession.version
1	NC_000001.11
2	NC_000002.12
3	NC_000003.12
...
X	NC_000023.11
Y	NC_000024.10
    \end{minted}
    \caption{\label{fig:chr2acc} chr2acc file that maps standard
      chromosome names to NCBI accession identifiers. Guidescan2 uses
      these identifiers internally since they uniquely identify
      organism version and chromosome simultaneously. }
\end{figure}

Now, since the reference genome is a raw FASTA file, we can run
Guidescan2 indexing on it as follows. Note that this step can require
up to 32GB of memory for large genomes. When generating the index for
\textit{hg38}, this step took a maximum of 30GB of memory. Large
temporary files with extension \texttt{.sdsl} will be created and
subsequently deleted during this process.

\begin{minted}[frame=lines]{shell-session}
  $ guidescan index --index hg38 GCF_000001405.26_GRCh38_genomic.fna
  $ rm GCF_000001405.26_GRCh38_genomic.fna.*.dna
\end{minted}

Note that we specify the prefix to ensure the commands are short and
delete the temporary files \texttt{*.dna} once the index has been
constructed. If memory constraints are an issue, we remind the user
that pre-constructed indices can be found \hl{here}.

With the index constructed, we then turn to processing the gRNAs into
the {\it kmer set} format for processing by Guidescan2. For the sake
of example, we will randomly select of 250 genes to process.

\begin{minted}[frame=lines]{shell-session}
  $ awk -F, '(NR > 1) { print $1 }' examples/sabatini_grnas.csv > id_list.txt
  $ cat id_list.txt | sed -e '/CTRL.*/d' | sed -r 's/sg(.*)_.*/\1/' > gene_list.txt
  $ uniq gene_list.txt | shuf | head -n 250 > random_genes.txt
  $ cat random_genes.txt | sed -r 's/.*/sg\0_/' > random_gene_ids.txt
\end{minted}

The preceding code grabs the \texttt{sgRNA ID} field of the CSV, drops
all control gRNAs, parses the gene, removes duplicates, and then
finally selects a random subset of genes. We include this as an
illustrative example of the custom processing that may need to
performed in order to analyze your gRNA set; it differs from dataset
to dataset.

With all the genes we want to analyze selected, we finally construct
our {\it kmers set}. We do this by selecting all gRNAs in our gene
set, sorting the rows, appending a PAM column, and adding in
our header. We also delete the temporary files that are no longer
needed.

\begin{minted}[frame=lines]{shell-session}
  $ grep -f random_gene_ids.txt examples/sabatini_grnas.csv > grnas.csv
  $ awk -F, '{ print $1 "," $6 ",NGG," $3","$4","$5 }' grnas.csv > kmers.csv
  $ sed -i '1i id,sequence,pam,chromosome,position,sense' kmers.csv
  $ rm id_list.txt gene_list.txt random_genes.txt random_gene_ids.txt grnas.csv
\end{minted}

At this point, it is convenient (though not necessary) to update the
chromosome names with NCBI accessions. As an example, we include the
following Python script \texttt{examples/chr2acc.py} that will perform
this update (Figure \ref{fig:chr2accpy}). We do not include a general
solution to this problem since it depends on the format of the gRNA
given to analyze, but for the example simply execute the following.

\begin{minted}[frame=lines]{shell-session}
  $ python examples/chr2acc.py chr2acc.txt kmers.csv > kmers2.csv
  $ mv kmers2.csv kmers.csv
\end{minted}

\begin{figure}[ht]
  \centering
  \begin{minted}[frame=lines]{python}
import sys

if len(sys.argv) < 3:
    print("usage: python chr2acc.py [chr2acc.txt] [kmers.csv]")
    sys.exit(1)

chr2acc = {}
with open(sys.argv[1]) as f:
    next(f)
    for l in f:
        words = l.split() 
        chr2acc[words[0]] = words[1]

with open(sys.argv[2]) as f:
    print(next(f), end='')
    for l in f:
        words = l.split(',')
        words[3] = chr2acc[words[3][3:]] # strips 'chr' prefix
        print(','.join(words), end='')
  \end{minted}
  \caption{\label{fig:chr2accpy} An example script for mapping
    chromosome to accession names.}
\end{figure}

Finally, our {\it kmer set} is ready to run through Guidescan2. After
moving everything to the correct directory, we can evaluate our {\it
  kmer set} with the following command.  \vspace{-0.8em}
\begin{minted}[frame=lines]{shell-session}
  $ guidescan enumerate ~/ncbi-*/hg38 -f kmers.csv -n 1 -o db.sam -a NAG
\end{minted}
As an estimate of time, this process took less than 10 minutes on a
Thinkpad t490. Since kmers are evaluated synchronously and written in
real time, progress can be measured via the number of lines in the
output database. That is,
\vspace{-0.8em}
\begin{minted}[frame=lines]{shell-session}
  $ wc -l db.sam
\end{minted}
tells you how many kmers have successfully been processed by Guidescan2.

To make our output human readable, we run our decoding script on the
output database, passing in the reference FASTA file as input.
\vspace{-0.8em}
\begin{minted}[frame=lines]{shell-session}
  $ python scripts/decode_database.py db.sam ~/ncbi-*/*.fna > processed.csv
\end{minted}

Finally, we are done! To enable the analysis of large sets of kmers,
say on the order of $10^6$ kmers, we recommend either reducing the
mismatch to 2 or parallelizing across several compute nodes. The
following sections on off-target database construction describe simple
strategies for parallelization.

\subsection{Off-Target Database Construction}
\label{subsec:offtargetdb}

Here we describe a program to construct genome-wide off-target
databases for organisms. To ensure this example runs quickly, we will
construct a databse for the C. elegans genome. An in particular, we
will use the RefSeq version provided by the National Center for
Biotechnology Information at the address:

\begin{center}
  \texttt{https://www.ncbi.nlm.nih.gov/assembly/GCF\_000002985.6/}
\end{center}

Once downloaded the reference genome must be decompressed into a raw
FASTA file.  To do this one can run the following sequence of
commands, assuming the downloaded tarball has name \texttt{cell.tar}.

\begin{minted}[frame=lines]{shell-session}
  $ tar -xvf cell.tar
  $ cd ncbi-genomes-2022-tarball
  $ gunzip GCF_000002985.6_WBcel235_genomic.fna.gz
\end{minted}

Now, since the reference genome is a raw FASTA file, we can run
Guidescan2 indexing on it as follows. Note that this step can require
up to 32GB of memory for large genomes. When generating the index for
\textit{hg38}, this step took a maximum of 30GB of memory. Large
temporary files with extension \texttt{.sdsl} will be created and
subsequently deleted during this process.

\begin{minted}[frame=lines]{shell-session}
  $ guidescan index --index cell.index GCF_000002985.6_WBcel235_genomic.fna
  $ rm GCF_000002985.6_WBcel235_genomic.fna.*.dna
\end{minted}

Note that we specify the prefix to ensure the commands are short and
delete the temporary files \texttt{*.dna} once the index has been
constructed. If memory constraints are an issue, we remind the user
that pre-constructed indices can be found \hl{here}.

At this point, we will deviate from the analysis in the previous
section. Instead of looking at a targeted set of gRNAs received from a
third-party, we will construct a genome wide set of gRNAs using
Guidescan2. To do this we need to specify the {\it targetable space}
of the genome. For standard CRISPR Cas9 enzymes, this corresponds to
the set of all 20-mers followed by NGG PAMs; but it varies across
CRISPR systems. Accordingly, we run the \texttt{generate\_kmers.py}
script with our desired parameters.

\begin{minted}[frame=lines]{shell-session}
  $ python scripts/generate_kmers.py --pam NGG --prefix "example-"\
    cell.fna > cell.kmers
\end{minted}

To enable parallel execution, we partition our kmer set across 100
pieces. The *nix \texttt{split} utility is useful for this
purpose. And then we append the header to each split. Since this is a
little bit more complex, we include it as a script (Figure
\ref{fig:splitkmers}). With the kmers set split into pieces, we can
run Guidescan2 seperately on each piece and then merge the resulting
SAM files together. The exact steps necessary depend upon on the
computing resources available, but it should be straightforward. As an
example, we will enumerate gRNAs in a sequential fashion.

\begin{minted}[frame=lines]{shell-session}
  $ for f in cell.*.csv; do\
  guidescan enumerate ce11.index -f $f -n 1 -o $f.sam -a NAG;\
  done
\end{minted}

Note that this step would take a \textbf{very} long time to complete
if run sequentially. For the example here, it would take around 1,000
minutes running on a single core on a Thinkpad t490; the time would
increase exponentially for organisms with longer genomes. Assuming
this step is completed, we finish database construction by merging the
databases together. To do this, we strip the headers from all but one
of the files and concatenate them together. This can be done using
samtools, but here we will just use \texttt{sed} and \texttt{cat}.

\begin{minted}[frame=lines]{shell-session}
  $ cp cell.000.sam cell.sam
  $ sed -E "/@.*/d" cell.*{1..99}.csv.sam >> cell.sam
\end{minted}

And at this step, database generation is complete. For storage, we
recommend compressing the file to a BAM format using Samtools. To make
our output human readable, we can run our decoding script on the
output database, or subsets thereof, passing in the reference FASTA
file as input. Remember to keep the header when splitting the database
into subsets. It can also be convenient to perform this step on the
individual database splits in parallel, though this step is very
quick.

\vspace{-0.8em}
\begin{minted}[frame=lines, breaklines]{shell-session}
  $ python scripts/decode_database.py cell.sam\ GCF_000002985.6_WBcel235_genomic.fna > ce11.csv
\end{minted}

\begin{figure}[ht]
  \centering
\begin{minted}[frame=lines]{bash}
#!/bin/bash

split -d -n l/100 -a 3 --additional-suffix .csv cell.kmers cell.
for f in cell.*.csv; do
    awk 'BEGIN {print "id,sequence,pam,chromosome,position,sense"} {print $1}'\
        $f > $f.header
    mv $f.header $f
done

# first file contains two headers now; we remove
tail -n +2 cell.000.csv > tmp.csv
mv tmp.csv cell.000.csv
\end{minted}
\caption{\label{fig:splitkmers} Bash script to split
  \texttt{cell.kmers} file into 100 partitions for parallelization.}
\end{figure}

\subsection{Off-Target Database Construction for F1-Cross Genomes}

To construct databases for hybrid genomes and enable allele specific
targeting, we need reference sequences for both alleles.  In our case
of generating allele specific guides for an F1 hybrid B6/129S1 mouse
genome, we used the \textit{mm38} reference genome and a synthetic
\textit{129S1} genome. If high quality genomes and mappings between them are
available for both alleles, then it is not necessary to generate a
synthetic genome. But this was not the case for 129S1.

We co-opted the MMARGE tool built for epigenetic data to construct
both a \textit{129S1} reference genome and a mapping between
\textit{129S1} and \textit{mm38} \parencite{link2018mmarge}. We used
the reference assembly \texttt{GRCm38\_68.fa} from Ensemble release 68
and the sequence variation files from the Mouse Genome Project
\parencite{keane2011mouse}. At the time of publication, the VCF files
are available for installation at:

\begin{center}
  \texttt{ftp://ftp-mouse.sanger.ac.uk}
\end{center}

Essentially, we pulled down the set of indels/SNPs as a VCF file and
processed them for usage with MMARGE. We also split the genome into
chromosomes, since this is required for processing. We refer the
reader to \parencite{link2018mmarge} for full details on constructing
synthentic genomes with MMARGE since it is rather detailed. However,
we include the code we used for this in (Listing
\ref{listing:mmarge}).

At this point, we assume the reader has two reference genomes,
referred to as \texttt{grcm38.fa} and \texttt{129s1.fa} and a mapping
between the coordinates of the two. We now run Guidescan2 off-target
database construction as in sub-section \ref{subsec:offtargetdb} for
each of these genomes seperately. This will result in two Guidescan2
databases \texttt{grcm38.sam} and \texttt{129s1.sam} for the
corresponding references. These consist of the sets of gRNAs that have
passed standard Guidescan2 filtering. From these databases, we
construct two {\it kmer sets} and then run these against the indices
for the opposite allele with a an exact match threshold of 0. This
will result in two sets of gRNAs that have exact matches on only one
allele, as well as complete off-target information for both alleles.

%\section{Miscellaneous Features}

\newpage
\begin{minted}[frame=lines, breaklines]{bash}
DIR=inputs/
MMARGE=MMARGE.pl
SNPS=$DIR/129S1_SvImJ.mgp.v5.snps.dbSNP142.vcf
INDELS=$DIR/129S1_SvImJ.mgp.v5.indels.dbSNP142.normed.vcf

mkdir -p $DIR

if [ ! -f $SNPS ]; then
    echo "Downloading and pre-processing latest SNP files for $(basename $SNPS)."
    echo "======================================================================"
    wget ftp://ftp-mouse.sanger.ac.uk/REL-1505-SNPs_Indels/strain_specific_vcfs/$(basename "$SNPS").gz -O "$SNPS".gz
    gunzip "$SNPS".gz
    grep "^#" $SNPS > "$SNPS".sorted
    grep -v "^#" $SNPS | sort -k1,1 -k2,2n --parallel=24 >> "$SNPS".sorted
fi

if [ ! -f $INDELS ]; then
    echo "Downloading and pre-processing latest INDEL files for $(basename $INDELS)."
    echo "======================================================================"
    wget ftp://ftp-mouse.sanger.ac.uk/REL-1505-SNPs_Indels/strain_specific_vcfs/$(basename "$INDELS").gz -O "$INDELS".gz
    gunzip "$INDELS".gz
    grep "^#" $INDELS > "$INDELS".sorted
    grep -v "^#" $INDELS | sort -k1,1 -k2,2n --parallel=24 >> "$INDELS".sorted
fi

REF_FTP=$(head $INDELS | PERL_BADLANG=0 perl -Xne 'print $1 if /reference=(ftp.*)/')
REF=$DIR/$(basename $REF_FTP)
if [ ! -f $REF ]; then
    echo "Installing reference genome from: $REF_FTP"
    echo "======================================================================"
    wget "$REF_FTP".gz -O $REF.gz
    gunzip $REF.gz
fi

echo "Splitting genome into chromosomes."
echo "======================================================================"

if [ ! -f scripts/faSplit ]; then
    wget http://hgdownload.soe.ucsc.edu/admin/exe/linux.x86_64.v385/faSplit
    chmod u+x scripts/faSplit
fi

mkdir -p genomes
./scripts/faSplit byname $REF genomes/

for f in genomes/*; do
    mv $f genomes/chr$(basename $f)
done

echo "Preparing files with MMARGE"
echo "======================================================================"

$MMARGE prepare_files -core 8 -files "$SNPS".sorted "$INDELS.sorted" -genome genomes -dir . -genome_dir . 
\end{minted}
\captionof{listing}{\label{listing:mmarge} Example code to construct to synthetic
  \textit{129S1} genome using MMARGE, including pre-processing steps.}

\newpage
\section{Bibliography}
\printbibliography[heading=none]

\end{document}
